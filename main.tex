\documentclass[12pt]{article}
\usepackage[utf8]{inputenc}
\usepackage[english,russian]{babel}
\usepackage{color}
\usepackage{listings}
\usepackage{caption}
\DeclareCaptionFont{white}{\color{white}}
\captionsetup[lstlisting]{format=listing,labelfont=white,textfont=white}

\title{Code Style}
\author{ИА-031 Сысюк З. Н.}
\date{Февраль 2022}

\begin{document}
\lstset{language=C, 
numbers=left,
basicstyle=\small\sffamily}

\maketitle

\section{Введение}
В документе содержится описание моего кодстайла

\section{Основная часть}
\textbf{Используемые языки программирования}
\begin{itemize}
    \item C/C++
\end{itemize}
\textbf{Пробелы и отступы}

\\Операторы и операнды разделяются пробелом:

\begin{lstlisting}
int x = ( a + b ) * c / d;
\end{lstlisting}

\\Так же пробелом отделяются и фигурные скобки:

\begin{lstlisting}
if (a == 5) { return; }
\end{lstlisting}
\textbf{Оформление циклов и операторов управления}

При использовании циклов или операторов управления используются отступы и переходы на новую строку, если это нужно:

\begin{lstlisting}
while(1) { 
    if(a == 2) {
        return;
    }
    a++;
}
\end{lstlisting}
\textbf{Разделение функций и блоков кода}

Функции и разные по смыслу блоки кода разделяются пустой строкой:
\begin{lstlisting}
int addFirst(...) {
....
}

int addLast(...) {
...
}
\end{lstlisting}
\textbf{Названия функций и переменных}

Названия функций и переменных должны быть логичными, не однобуквенными (названия итераторов могут состоять из одной буквы:
\begin{lstlisting}
int c; // BAD
int counter; // GOOD
\end{lstlisting}

Названия функций должны быть записаны в смешанном регистре и начинаться с нижнего:
\begin{lstlisting}
int addFirst(...) {
....
}

int addLast(...) {
...
}
\end{lstlisting}

Именнованые константы должны быть записаны в верхнем регистре с нижним подчеркивание в качестве разделителя:
\begin{lstlisting}
const int MAX_SIZE = 1000;
const int N = 10;
\end{lstlisting}

\textbf{Лучшие практики}

Используйте текстовую строку, стандартную для C++, а не С. С++ путает тем, что имеет два вида текстовых строк: класс string из С++ и старый char* (массив символов) из С.
\begin{lstlisting}
// Bad practice
char* str = "Hello there";

// Good practice
string str = "Hello there";
\end{lstlisting}

С++ основан на С, поэтому всегда есть вариант решить задачу «путем С++» и «путем С». Например, когда вы желаете вывести что-либо на системную консоль, вы можете сделать это «путем С++» , использовав оператор вывода cout, в то время как «путем С» вы бы использовали глобальную функцию вроде printf.
\begin{lstlisting}
// Bad practice
printf("Hello, world!\n");
// Good practice
cout << "Hello, world!" << endl;
\end{lstlisting}
\textbf{Файлы исходных кодов}

 Класс следует объявлять в заголовочном файле и определять (реализовывать) в файле исходного кода, имена файлов совпадают с именем класса:
\begin{lstlisting}
myClass.h myClass.cpp
\end{lstlisting}

Все определения должны находиться в файлах исходного кода:
\begin{lstlisting}
class MyClass
{
public:
  int getValue () {return value_;}  // WRONG
  ...

private:
  int value_;
}
\end{lstlisting}
Заголовочные файлы объявляют интерфейс, файлы исходного кода его реализовывают.
\\Не следует объявлять переменные класса как public. Вместо этого нужно использовать переменные с модификатором private и соответствующие
функции доступа.

\section{Вывод}
В ходе работы я описал свой стиль написания кода в редакторе \textit{LATEX}.

\section{Список литературы}

\begin{enumerate}
    \item https://tproger.ru/translations/stanford-cpp-style-guide/
    \item https://habr.com/ru/post/172091/
    \item https://habr.com/ru/company/ruvds/blog/574352/
\end{enumerate}
\end{document}
